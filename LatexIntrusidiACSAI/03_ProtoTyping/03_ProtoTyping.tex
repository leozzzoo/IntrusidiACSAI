\chapter{PaperPrototyping}

\section{Prototipo cartaceo 1}
Nella progettazione del primo prototipo cartaceo, abbiamo prestato particolare attenzione al fatto che tutti e tre i task selezionati
siano "task di ricerca", ognuno con diverse macro-aree oggetto delle ricerche. Pertanto, abbiamo cercato di strutturare l'applicazione
in tre sezioni accessibili dalla schermata iniziale tramite tre tasti denominati "Corsi di Laurea", "Insegnamenti" e "Professori", ciascuno
dei quali reindirizza alla relativa schermata contenente esclusivamente la barra di ricerca.
Il nostro principale obiettivo nella progettazione è quello di condensare informazioni tematiche in un'unica schermata. Dopo le schermate
di ricerca si accede, infatti, alla schermata specifica del soggetto ricercato: un professore, un insegnamento (con relativo professore e
anno accademico), o un corso di laurea specifico, seguendo la suddivisione precedentemente presentata. Ad esempio, per un dato insegnamento,
questa schermata mostrerà una serie di tasti quali "Appunti Lezioni", "Modalità d'esame", "Opinioni studenti" e altri, permettendo all'utente
di avere le informazioni chiave, riguardo il soggetto ricercato, tutte a portata di mano e facili da distinguere.


\subsection{Valutazione con l'esperto}
Durante la valutazione condotta dal membro del nostro gruppo scelto per essere l'esperto è emersa una violazione critica di un'euristica di
usabilità: la mancanza di un metodo chiaramente contrassegnato per tornare indietro alla schermata precedente.
Inoltre, attraverso un cognitive walkthrough, l'esperto ha suggerito l'inserimento delle "ricerche recenti" nelle schermate di ricerca per ottimizzare le operazioni.
Nonostante questo abbiamo deciso di procedere con la valutazione con gli utenti senza modifiche al prototipo.


\subsection{Valutazione con gli utenti}
Per la valutazione sul campo con gli utenti abbiamo optato per la tecnica di osservazione think aloud per cui, dopo aver presentato lo scenario
e il task corrspondente al prototipo, l'utente è stato invitato a ragionare ad alta voce nello svolgere il task. È stato evidenziato come la mancanza
di un'indicazione chiara sulla posizione dell'utente nel flusso dell'applicazione possa generare confusione. È stata, poi, ulteriormente confermata la necessità di
un tasto per tornare indietro.
Inoltre, abbiamo notato un'esitazione da parte di alcuni utenti nel selezionare i tasti nelle schermate specifiche, per esempio quella dell'insegnamento descritta
prima. Questa incertezza potrebbe essere dovuta al cambiamento completo della schermata dopo la selezione di un tasto, portando gli utenti a riflettere di più prima della scelta.


\section{Prototipo cartaceo 2}
Per affrontare le criticità riscontrate nel prototipo precedente abbiamo introdotto un tasto per tornare indietro, insieme a degli indicatori testuali per chiarire
la posizione dell'utente all'interno dell'applicazione.
Per ridurre l'incertezza degli utenti riscontrata precedentemente, abbiamo introdotto, dove il cambio di schermata risultava superfluo, tasti a tendina. Questi,
infatti, risultano meno invasivi, facilitando la selezione delle opzioni senza implicare un cambiamento immediato della schermata. Ad esempio, selezionando su
"Appunti Lezioni" non viene più aperta una nuova schermata ma appare una tendina, sotto il tasto selezionato, con una lista di argomenti tra cui gli utenti possono
scegliere quale cliccare per essere poi portati alla schermata in cui sono presenti gli appunti desiderati.
Seguendo il suggerimento dell'esperto, sono anche state aggiunte le ricerche recenti in tutte le schermate di ricerca.


\subsection{Valutazione con l'esperto}
Durante la valutazione, basata sulle euristiche, fatta dall’esperto non sono state trovate particolari violazioni delle linee guida. L’esperto crede che l’utente non avrà difficoltà nello svolgere i task.
Per questo abbiamo deciso di passare direttamente alla valutazione con gli utenti.


\subsection{Valutazione con gli utenti}
Come metodo di valutazione del secondo paper prototype. abbiamo optato per la stessa tecnica utilizzata precedentemente, ovvero il think aloud. Generalmente, non
sono state trovate particolari criticità da parte degli utenti durante l’esecuzione dei task. L’unica funzionalità risultata mancante da parte dell’utente è quella
dell’ordinamento delle opinioni e degli appunti in base a criteri specifici, come la valutazione assegnata da altri studenti o la data di pubblicazione. Per questo
abbiamo deciso di inserire questa funzionalità direttamente nel digital prototype realizzato successivamente, per poi procedere alla valutazione dello stesso.